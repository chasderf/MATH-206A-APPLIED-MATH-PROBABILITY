\documentclass{article}
\usepackage{amsmath}
\usepackage{amssymb}
\newcommand\tab[1][1cm]{\hspace*{#1}}
\begin{document}

\title{Math 206A Probability: Chapter 3 Central Limit Theorems written by Rick Durrett}
\author{This article is transcribed by Charlie Seager}

\maketitle

\textbf {Chapter 3.1 The De Moivre-Laplace Theorem}

\textbf {Lemma 3.1.1} If $c_j \rightarrow 0, a_j \rightarrow \infty$ and $a_j c_j \rightarrow \lambda$ then $(1 + c_j)^{aj} \rightarrow e^{\lambda}$.

\textbf {Theorem 3.1.2} If 2k/ $\sqrt{2n} \rightarrow x$ then $P(S_{2n} = 2k) ~ (\pi n)^{-1/2} e^{{-x^2}/2}.$

\textbf {Theorem 3.1.3 The De Moivre-Laplace Theorem} If $a < b$ then as $m \rightarrow \infty$
\begin{center}
$P(a \leq S_m / \sqrt{m} \leq b) \rightarrow \int_{a}^{b} (2\pi)^{-1/2} e^{{-x^2}/2}dx$
\end{center}

\textbf{Chapter 3.2 Weak Convergence}

\textbf {Lemma 3.2.7} $V_{n+1}$ has density function
\begin{center}
$f v_{n+1} (x) = (2n + 1) (^{2n}_n) x^n (1-x)^n$
\end{center}

\textbf {Chapter 3.2.2 Theory}

\textbf{Theorem 3.2.8} If $F_n \Rightarrow F_\infty$ then there are random variables $Y_n, 1 \leq n \leq \infty$ with distribution $F_n$ so that $Y_n \rightarrow Y_\infty$ a.s.

\textbf {Theorem 3.2.9} $X_n \Rightarrow X_\infty$ if and only if for every bounded continous function g we have $Eg(X_n) \rightarrow Eg(X_\infty)$

\textbf {Theorem 3.2.10 Continous mapping theorem} Let g be a measurable function and $D_g = \{x : g$ is discontinous at x\}. If $X_n \Rightarrow X_\infty$ and $P(X_\infty \in D_g) = 0$ then $g(X_n) \Rightarrow g(X)$. If in addition g is bounded then $Eg(X_n) \rightarrow Eg(X_\infty)$

\textbf {Theorem 3.2.11} The following statements are equivalent: (i) $X_n \Rightarrow X_\infty$ \\
(ii) For all open sets G, lim $inf_{n \to \infty} P(X_n \in G) \geq P(X_\infty \in G)$ \\
(iii) For all closed sets K, lim $sup_{n \to \infty} P(X_n \in K) \leq P(X_\infty \in K)$ \\
(iv) For all Borel sets A with $P(X_\infty \in \partial A) = 0, lim_{n \to \infty} P(X_n \in A) = P(X_\infty \in A).$

\textbf {Theorem 3.2.12 Helly's selection theorem} For every sequence $F_n$ of distribution functions, there is a subsequence $F_{n(k)}$ and a right continuous nondecreasing function F so that $lim_{k \to \infty} F_{n(k)} (y) = F(y)$ at all continuity points y of F.

\textbf {Theorem 3.2.13} Every subsequential limit is the distribution function of a probability measure if and only if the sequence $F_n$ is tight, i.e. for all $\epsilon > 0$ there is an M, so that
\begin{center}
lim$sup_{n\to \infty}1 - F_n (M_\epsilon) + F_n (-M_\epsilon) \leq \epsilon$
\end{center}

\textbf {Theorem 3.2.14} If there is a $\varphi \geq 0$ so that $\varphi (x) \to \infty$ as $|x| \to \infty$ and 
\begin{center}
$C = sup_n \int \varphi(x) dF_n (x) < \infty$
\end{center}

\textbf {Theorem 3.2.15} If each subsequence of $X_n$ has a further subsequence that converges to X then $X_n \Rightarrow X$.

\textbf {Chapter 3.3 Characteristic Functions}

\textbf {Chapter 3.3.1 Definition, Inversion Formula}

\textbf {Theorem 3.3.1} All characteristic functions have the following properties: \\
(a) $\varphi(0) = 1 $ \\
(b) $\varphi(-t) = \bar{\varphi(t)}$ \\
(c) $| \varphi(t)| = |Ee^{itX}| \leq E|e^{itX}| = 1$ \\ 
(d) $| \varphi(t + h) - \varphi(t)| \leq E|e^{ihX}-1|,$ so $\varphi(t)$ is uniformly continous on $(-\infty, \infty)$ \\
(e) $E e^{it(aX + b)} = e^{itb} \varphi(at)$

\textbf {Theorem 3.3.2} If $X_1$ and $X_2$ are independent and have ch.f.'s $\varphi_1$ and $\varphi_2$ then $X_1 + X_2$ has ch.f. $\varphi_1(t) \varphi_2(t)$

\textbf {Lemma 3.3.9} If $F_1 ,..., F_n$ have ch.f. $\varphi_1 ,..., \varphi_n$ and $\lambda_i \geq 0$ have $\lambda_1 + \dots + \lambda_n = 1$ then $\sum_{i=1}^n \lambda_i F_i$ has ch.f. $\sum_{i=1}^n \lambda_i \varphi_i$

\textbf {Theorem 3.3.11 The inversion formula} Let $\varphi(t) = \int e^{itx} \mu (dx)$ where $\mu$ is a probability measure. If $a < b$ then
\begin{center}
$lim_{T \to \infty} (2 \pi)^{-1} \int_{-T}^{T} \frac{e^{-ita} - e^{-itb}}{it} \varphi(t) dt = \mu (a,b) + \frac{1}{2} \mu (\{a,b\})$
\end{center}

\textbf {Corollary 3.3.12} If $\varphi$ is real then X and -X have the same distribution. 

\textbf {Corollary 3.3.13} If $X_i$ i =1,2 are independent and have normal distributions with mean 0 and variance $\sigma_i^2$ then $X_1 + X_2$ has a normal distribution with mean 0 and variance $\sigma_1^2 +\sigma_2^2$

\textbf {Theorem 3.3.14} If $\int |\varphi(t)|dt < \infty$ then $\mu$ has bounded continous density
\begin{center}
$f(y) = \frac{1}{2\pi} \int e^{-ity} \varphi(t) dt$
\end{center}

\textbf {Chapter 3.3.2 Weak Convergence}

\textbf {Theorem 3.3.17 Continuity theorem} Let $\mu_n, 1 \leq n \leq \infty$ be probability measures with ch.f. $\varphi_n$. \\
(i) If $\mu_n \Rightarrow \mu_\infty$ then $\varphi_n(t) \rightarrow \varphi_\infty(t)$ for all t. \\
(ii) If $\varphi_n(t)$ converges pointwise to a limit $\varphi(t)$ that is continuous at 0, then the associated sequence of distributions $\mu_n$ is tight and converges to the measure $\mu$ with characteristic function $\varphi$

\textbf {Chapter 3.3.3 Moments and Derivatives}

\textbf {Theorem 3.3.18} If $\int |x|^n \mu(dx) < \infty$ then its characteristic function $\varphi$ has a continuous derivative of order n given by $\varphi^{(n)}(t) = \int (ix)^n e^{itx} \mu (dx)$

\textbf {Lemma 3.3.19} 
\begin{center}
$|e^{ix} - \sum_{m=0}^n \frac{(ix)^m}{m!} | \leq min (\frac{|x|^{n+1}}{(n+1)!^{'}} \frac{2|x|^n}{n!} )$
\end{center}
The first term on the right is the usual order of magnitude we expect in the correction term. The second is better for large $|x|$ and will help us prove the central limit theorem without assuming finite third moments.

\textbf {Theorem 3.3.20} If $E|X|^2 < \infty$ then 
\begin{center}
$\varphi(t) = 1 + itEX - t^2 E(X^2) / 2 + o(t^2)$
\end{center}

\textbf {Theorem 3.3.21} If lim$sup_{h|0} \{ \varphi(h) - 2\varphi(0) + \varphi(-h)\} / h^2 > -\infty$ then $E|X|^2 < \infty$

\textbf {Chapter 3.3.4 Polya's Criterion*}

\textbf {Theorem 3.3.22 Polya's criterion} Let $\varphi(t)$ be real nonnegative and have $\varphi(0) = 1, \varphi(t) = \varphi(-t)$ and $\varphi$ is decreasing and convex on $(0, \infty)$ with
\begin{center}
$lim_{t \downarrow 0} \varphi(t) = 1, \tab lim_{t \uparrow \infty} \varphi(t) = 0$
\end{center}
Then there is a probability measure v on $(0, \infty)$, so that
(*) \begin{center}
$\varphi(t) = \int_{0}^{\infty} (1 - | \frac{t}{s}|)^+ v(ds)$
\end{center}
and hence $\varphi$ is a characteristic function.

\textbf {Chapter 3.3.5 The Moment Problem*}

\textbf {Theorem 3.3.25} If lim$sup_{k \to \infty} \mu_{2k}^{1/2k} / 2k = r < \infty$ then there is at most one d.f. F with $\mu_k = \int x^k dF(x)$ for all positive integers k.

\textbf {Theorem 3.3.26} Suppose $\int x^k dF_n(x)$ has a limit $\mu_k$ for each k and 
\begin{center}
lim$sup_{k \to \infty} \mu_2k^{1/2k} / 2k < \infty$
\end{center}
then $F_n$ converges weakly to the unique distribution with these moments.

\textbf {Chapter 3.4 Central Limit Theorems}

\textbf {Chapter 3.4.1 i.i.d. Sequences}

\textbf {Theorem 3.4.1} Let $X_1 , X_2 ,...$ be i.id. with $EX_i = \mu$, $var(X_i) = \sigma^2 \in (0, \infty)$. If $S_n = X_1 + \dots + X_n$ then 
\begin{center}
$(S_n - n \mu) / \sigma n^{1/2} \Rightarrow \mathcal{X}$
\end{center}
where $\mathcal{X}$ has the standard normal distribution.

\textbf {Theorem 3.4.2} If $c_n \rightarrow c \in C$ then $(1 + c_n /n)^n \rightarrow e^c$

\textbf {Lemma 3.4.3} Let $z_1 ,..., z_n$ and $w_1 ,.., w_n$ be complex numbers of modulus $\leq \theta$. Then
\begin{center}
$| \prod_{m=1}^n z_m - \prod_{m=1}^n w_m | \leq \theta^{n-1} \sum_{m=1}^{n} |z_m - w_m|$
\end{center}

\textbf {Lemma 3.4.4} If b is a complex number with $|b| \leq 1$ then $|e^b - (1 + b)| \leq |b|^2$.

\textbf {Chapter 3.4.2 Triangular Arrays}

\textbf {Theorem 3.4.10 The Lindeberg-Feller Theorem} For each n, let $X_{n,m}, 1 \leq m \leq n$, be independent random variables with $EX_{n,m} = 0$. Suppose \\
(i) $\sum_{m=1}^n EX_{n,m}^2 \rightarrow \sigma^2 > 0$ \\
(ii) For all $\epsilon > 0,$ $lim_{n \to \infty} \sum_{m=1}^n E(|X_{n,m}|^2: |X_{n,m}| > \epsilon) = 0$ \\
Then $S_n = X_{n,1} + \dots + X_{n,n} \Rightarrow \sigma \mathcal{X}$ as $n \rightarrow \infty$

\textbf {THeorem 3.4.14} Let $X_1, X_2 ,...$ be i.i.d. and $S_n = X_1 + \dots + X_n$. In order that there exist constants $a_n$ and $b_n > 0$ so that $(S_n - a_n) / b_n \Rightarrow \mathcal{X}$ it is necessary and sufficient that
\begin{center}
$y^2 P(|X_1| > y) / E(|X_1|^2; |X_1| \leq y) \rightarrow 0$
\end{center}

\textbf {Chapter 3.4.3 Prime Divisors (Erdos-Kac)*}

\textbf {Lemma 3.4.15} $h_n(\epsilon) \rightarrow 0$ for each fixed $\epsilon > 0$ so we can pick $\epsilon_n \rightarrow 0$ so that $h_n(\epsilon_n) \rightarrow 0$

\textbf {Theorem 3.4.16 Erdos-Kac central limit theorem} As $n \rightarrow \infty$
\begin{center}
$P_n(m \leq n : g(m) - log log n \leq x(log log n)^{1/2}) \rightarrow P(\mathcal{X} \leq x)$
\end{center}

\textbf {Chapter 3.4.4 Rates of Convergence (Berry-Esseen)*}

\textbf{Theorem 3.4.17} Let $X_1, X_2 ,...$ be i.i.d. with $EX_i = 0, EX_i^2 = \sigma^2$, and $E|X_i|^3 = p < \infty$. If $F_n(x)$ is the distribution of $(X_1 + \dots + X_n) / \sigma \sqrt{n}$ and $\mathcal{N}(x)$ is the standard normal distribution, then
\begin{center}
$|F_n(x) - \mathcal{N} (x) | \leq 3p / \sigma^3 \sqrt{n}$
\end{center}

\textbf {Lemma 3.4.18} Let F and G be distribution functions with $G^{'}(x) \leq \lambda < \infty.$ Let $\Delta(x) = F(x) - G(x),$  $\mathfrak{N} = sup|\Delta(x)|, \Delta_L = \Delta * H_L$, and $\mathfrak{N}_L = sup|\Delta_L (x)|.$ Then

\begin{center}
$\mathfrak{N}_L \geq \frac{\mathcal{N}}{2} - \frac{12 \lambda}{\pi L}$ \tab or \tab $\mathcal{N} \leq 2\mathcal{N}_L + \frac{24 \lambda}{\pi L}$
\end{center}

\textbf {Lemma 3.4.19} Let $K_1$ and $K_2$ be d.f. with mean 0 whose ch.f. $\mathcal{k}_i$ are integrable
\begin{center}
$K_1(x) - K_2(x) = (2\pi)^{-1} \int -e^{itx} \frac{\mathcal{k}_1(t) - \mathcal{k}_2(t)}{it} dt$
\end{center}

\textbf {Chapter 3.5 Local Limit Theorems*}

\textbf {Theorem 3.5.2} Let $\varphi(t) = Ee^{itX}$. There are only three possibilities \\
(i) $|\varphi(t)| < 1$ for all $t \neq 0$ \\
(ii) There is a $\lambda > 0$ so that $|\varphi(\lambda)| = 1$ and $|\varphi(t)| < 1$ for $0 < t < \lambda$. In this case, X has a lattice distribution with span $2\pi / \lambda$ \\
(iii) $|\varphi(t)| = 1$ for all t. In this case, X = b a.s. for some b.

\textbf {Theorem 3.5.3} Under the hypotheses above, as $n \to \infty$
\begin{center}
$sup_{x \in \mathcal{L}_n }|\frac{n^{1/2}}{h} p_n (x) - n(x)| \rightarrow 0$
\end{center}

\textbf {Theorem 3.5.4} Under the hypotheses above, if $x_n / \sqrt{n} \to x$ and $a < b$
\begin{center}
$\sqrt{n} P(S_n \in (x_n + a,$ $x_n + b)) \rightarrow (b-a)n(x)$
\end{center}

\textbf {Poisson Convergence}

\textbf {Chapter 3.6.1 The Basic Limit Theorem}

\textbf {Theorem 3.6.1} For each n let $X_{n,m}, 1 \leq m \leq n$ be independent random variables with $P(X_{n,m} = 1) = p_{n,m}, P(X_{n,m} = 0) = 1 - p_{n,m}.$ Suppose \\
(i) $\sum_{m=1}^n p_{n,m} \rightarrow \lambda \in (0, \infty)$, \\
and (ii) $max_{1 \leq m \leq n} p_{n,m} \rightarrow 0$
If $S_n = X_{n,1} + \dots X_{n,n}$ then $S_n \Rightarrow Z$ where Z is Poisson$(\lambda)$.

\textbf {Lemma 3.6.4} (i) $d(\mu, v) = ||\mu - v||$ defines a metric on probability measures on Z and \\
(ii) $||\mu_n - \mu|| \rightarrow 0$ if and only if $\mu_n (x) \rightarrow \mu(x)$ for each $x \in Z,$ which by Excercise 3.2.11 is equivalent to $\mu_n \Rightarrow \mu$

\textbf {Lemma 3.6.5} If $\mu_1 x \mu_2$ denotes the product measure on Z x Z that has $(\mu_1 x \mu_2)(x,y) = \mu_1(x) \mu_2(y)$ then
\begin{center}
$||\mu_1 x \mu_2 - v_1 x v_2|| \leq ||\mu_1 - v_1|| + ||\mu_2 - v_2||$
\end{center}

\textbf {Lemma 3.6.6} If $\mu_1 * \mu_2$ denotes the convolution of $\mu_1$ and $\mu_2$, that is
\begin{center}
$\mu_1 * \mu_2(x) = \sum_{y} \mu_1 (x-y) \mu_2 (y)$
\end{center}

\textbf {Lemma 3.6.7} Let $\mu$ be the measure with $\mu(1) = p$ and $\mu(0) = 1 - p$. Let v be a Poisson distribution with mean p. Then $||\mu - v|| \leq p^2$.

\textbf {Chapter 3.6.2 Two Examples with Dependence}

\textbf {Theorem 3.6.10} If $ne^{-r/n} \rightarrow \lambda \in [0, \infty)$ the number of empty boxes approaches a Poisson distribution with mean $\lambda$.

\textbf {Chapter 3.7 Poisson Processes}

\textbf {Theorem 3.7.1} Let $X_{n,m}, 1 \leq m \leq n$ be independent nonnegative integer valued random variables with $P(X_{n,m} = 1) = p_{n,m}, P(X_{n,m} \geq 2) = \epsilon_{n,m}$. \\
(i) $\sum_{m=1}^n p_{n,m} \rightarrow \lambda \in (0, \infty)$, \\
(ii) $max_{1 \leq m \leq n} p_{n,m} \rightarrow 0,$ \\
and (iii) $\sum_{m=1}^n \epsilon_{n,m} \rightarrow 0$ \\
If $S_n = X_{n,1} + \dots + X_{n,n}$ then $S_n \Rightarrow Z$ where Z is Poisson($\lambda$).

\textbf {Theorem 3.7.2} If (i)-(iv) hold then N(0,t) has a Poisson distribution with mean $\lambda$t.

\textbf {Chapter 3.7.1 Compound Poisson Procces}

\textbf {Theorem 3.7.3} Let $Y_1, Y_2 ,...$ be independent and identically distributed let N be an independent nonnegative integer valued random variable and let $S = Y_1 + \dots + Y_n$ with S=0 when N=0 \\
(i) If $E|Y_i|, EN < \infty$, then $ES = EN \cdot EY_i$ \\
(ii) If $EY_i^2, EN^2 < \infty$, then var(S) = EN$var(Y_i) + var(N)(EY_i)^2$ \\
(iii) If N is Poisson ($\lambda$), then var(S) = $\lambda EY_i^2$.

\textbf {Chapter 3.7.2 Thinning}

\textbf {Theorem 3.7.4} $N_j(t)$ are independent rate $\lambda P(Y_i = j)$ Poisson processes.

\textbf {Theorem 3.7.5} Suppose that a Poisson process with rate $\lambda$ we keep a point that lands at s with probability p(s). Then the result is a nonhomogeneous Poisson process with rate $\lambda $p(s).

\textbf {Chapter 3.7.3 Conditioning}

\textbf {Theorem 3.7.9} Let $T_n$ be the time of the nth arrival in a rate $\lambda$ Poisson process. Let $U_1 , U_2 ,..., U_n$ be independent uniform on (0,t) and let $V_k^n$ be the kth smallest number in $\{U_1 ,..., U_n\}$. If we condition on N(t) = n. Thevectors $V = (V_1^n ,\dots , V_n^n)$ and $T = (T_1 , \dots T_n)$ have the same distribution.

\textbf {Corollary 3.7.10} If $0 < s < t$ then 
\begin{center}
$P(N(s) = m|N(t) = n) = (^n_m) (\frac{s}{t})^m (1 - \frac{s}{t})^{n-m}$
\end{center}

\textbf {Theorem 3.7.11} Let $T_n$ be the time of the nth arrival in a rate $\lambda$ Poisson process. Let $U_1 , U_2 ,..., U_n$ be independent uniform on (0,1) and let $V_k^n$ be the kth smallest number in $\{U_1 ,..., U_n\}$. The vectors $(V_1^n ,\dots , V_n^n)$ and $(T_1 / T_{n+1} , \dots , T_n / T_{n+1})$ have the same distribution.

\textbf {Chapter 3.8 Stable Laws*}

\textbf {Lemma 3.8.1} If $h_n(\epsilon) \rightarrow g(\epsilon)$ for each $\epsilon > 0$ and $g(\epsilon) \rightarrow g(0)$ as $\epsilon \rightarrow 0$ then we can pick $\epsilon_n \rightarrow 0$ so that $h_n(\epsilon_n) \rightarrow g(0)$

\textbf {Theorem 3.8.2} Suppose $X_1 , X_2 ,...$ are i.i.d. with a distribution that satisfies \\
(i) $lim_{x \to \infty} P(X_1 > x) / P(|X_1| > x) = \theta \in [0,1]$ \\
(ii) $P(|X_1| > x) = x^{-\alpha} L(x)$ \\
where $\alpha < 2$ and L is slowly varying. Let $S_n = X_1 + \dots + X_n$ \\
$a_n = inf\{x : P(|X_1| > x) \leq n^{-1}\}$  and   $b_n = nE(X_1 1_{(|X_1| \leq a_n)})$ \\
As $n \to \infty, (S_n - b_n) / a_n \Rightarrow Y$ where Y has a nondegenerate distribution.

\textbf {Lemma 3.8.3} For any $\delta > 0$ there is C so that for all $t \geq t_0$ and $y \leq 1$ \\
\begin{center}
$P(|X_1|>yt)/P(|X_1| > t) \leq Cy^{-\alpha - \delta}$
\end{center}

\textbf {Theorem 3.8.8} Y is a limit of $(X_1 + \dots + X_k - b_k) /a_k$ for some i.i.d. sequence $X_i$ if and only if Y has a stable law.

\textbf {Theorem 3.8.9 Convergence of types theorem} If $W_n \Rightarrow W$ and there are constants $\alpha_n > 0, \beta_n$ so that $W_n^{'} = \alpha_n W_n + \beta_n \Rightarrow W^{'}$ where W and $W^{'}$ are nondegenerate, then there are constants $\alpha$ and $\beta$ so that $\alpha_n \rightarrow \alpha$ and $\beta_n \to \beta$

\textbf {Chapter 3.9 Infinitely Divisible Distributions*}

\textbf {Theorem 3.9.1} Z is a limit of sums of type (*) if and only if Z has an infinitely divisible distribution.

\textbf {Theorem 3.9.6 Levy-Khinchin Theorem} Z has an infinitely divisible distribution if and only if its characteristic function has 
\begin{center}
$log \varphi(t) = ict - \frac{\sigma^2 t^2}{2} + \int (e^{itx} - 1 - \frac{itx}{1+x^2}) \mu(dx)$
\end{center}
where $\mu$ is a measure with $\mu(\{0\}) = 0$ and $\int \frac{x^2}{1+x^2} \mu(dx) < \infty$

\textbf {Theorem 3.9.7 Kolmogorov's Theorem} Z has an infinitely divisible distribution with mean 0 and finite variance if and only if its ch.f. has 
\begin{center}
$log \varphi(t) = \int (e^{itx} - 1 - itx) x^{-2} v (dx)$
\end{center}
Here the integrand is $-t^2 / 2$ at 0, v is called the canonical measure and var(Z) = v(R).

\textbf {Chapter 3.10 Limit Theorems in $R^d$}

\textbf {Theorem 3.10.1} The following statements are equivalent:  \\
(i) $Ef(X_n) \to Ef(X_\infty)$ for all bounded continuous f. \\
(ii) $Ef(X_n) \to Ef(X_\infty)$ for all bounded Lipshitz continuous f. \\
(iii) For all closed sets K, $limsup_{n \to \infty} P(X_n \in K) \leq P(X_\infty \in K)$. \\
(iv) For all open sets G, $liminf_{n \to \infty} P(X_n \in G) \geq P(X_\infty \in G)$ \\
(v) For all sets A with $P(X_\infty \in \partial A) = 0,$ $lim_{n \to \infty} P(X_n \in A) = P(X_\infty \in A)$ \\
(vi) Let $D_f$ = the set of discontinuities of f. For all bounded functions f with $P(X_\infty \in D_f) = 0$ we have $Ef(X_n) \to Ef(X_\infty)$.

\textbf {Theorem 3.10.2} On $R^d$ weak convergence defined in terms of convergence of distribution $F_n \Rightarrow F$ is equivalent to notion of weak convergence defined for a general metric space.

\textbf {Theorem 3.10.3} If $\mu_n$ is tight, then there is a weakly convergent subsequence.

\textbf {Theorem 3.10.4 Inversion formula} if $A = [a_1 , b_1] x ... x [a_d, b_d]$ with $\mu(\partial A) = 0$ then
\begin{center}
$\mu(A) = lim_{T \to \infty} (2\pi)^{-d} \int_{[-T, T]^d} \prod_{j=1}^d \psi_j(t_j) \varphi(t) dt$
\end{center}
where $\psi_j(s) = (exp(-isa_j) -exp(-isb_j))/is.$

\textbf {Theorem 3.10.5 Convergence theorem} Let $X_n,$ $1 \leq n \leq \infty$ be random vectors with ch.f. $\varphi_n$. A necessary and sufficient condition for $X_n \Rightarrow X_\infty$ is that $\varphi_n(t) \to \varphi_\infty(t)$

\textbf {Theorem 3.10.6 Cramer Wold Device} A sufficient condition for $X_n \Rightarrow X_\infty$ is that $\theta \cdot X_n \Rightarrow \theta \cdot X_\infty$ for all $\theta \in R^d$

\textbf {Theorem 3.10.7 The central limit theorem in $R^d$} let $X_1, X_2 ,...$ be i.i.d. random vectors with $EX_n = \mu$ and finite covariances 
\begin{center}
$\Gamma_{i,j} = E((X_{n,j} - \mu_i)(X_{n,j} - \mu_j))$
\end{center}
If $S_n = X_1 + \dots + X_n$ then $(S_n - n\mu)/n^{1/2} \Rightarrow \mathcal{X}$ where $\mathcal{X}$ has a multivariate normal distribution with mean 0 and covariance $\Gamma$, i.e.
\begin{center}
$Eexp(i\theta \cdot \mathcal{X}) = exp (- \sum_i \sum_j \theta_i \theta_j \Gamma_{i,j} / 2)$
\end{center}























\end{document}