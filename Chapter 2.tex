\documentclass{article}
\usepackage{amsmath}
\newcommand\tab[1][1cm]{\hspace*{#1}}
\begin{document}

\title{Math 206A Probability: Chapter 2 Laws of Large Numbers written by Rick Durrett}
\author{This article is transcribed by Charlie Seager}

\maketitle

\textbf {Chapter 2.1 Independence}

\textbf {Theorem 2.1.1} \\ 
(i) If X and Y are independent then $\sigma(X)$ and $\sigma(Y)$ are \\
(ii) Conversely, if $\mathcal{F}$ and $\mathcal{G}$ are independent, $X \in \mathcal{F}$ and $Y \in \mathcal{G}$ then X and Y are independent.

\textbf {Theorem 2.1.2} (i) If A and B are independent then so are $A^c$ and B, A and $B^c$, and $A^c$ and $B^c$. \\ (ii) Conversely events A and B are independent if and only if their indicator random variables $1_A$ and $1_B$ are independent.

\textbf {Theorem 2.1.3} Let $A_1, A_2,..., A_n$ be independent. \\ (i) $A_1^{c}, A_2 ,..., A_n$ are independent; \\ (ii) $1_{A_1},..., 1_{A_n}$ are independent.

\textbf {Chapter 2.1.1 Sufficient Conditions for Independence}

\textbf {Lemma 2.1.5} Without loss of generality we can suppose $A_i$ contains $\Omega$. In this case the condition is equivalent to
\begin{center}
$P(\cap_{i=1}^n A_i) = \prod_{i=1}^{n} P(A_i) \tab$ whenever $A_i \in \mathcal{A}_{i}$
\end{center}
since we can set $A_i = \Omega$ for $i \not\in  I$.

\textbf {Theorem 2.1.6} $\pi - \lambda$ Theorem. If $\mathcal{P}$ is a $\pi-system$ and $\mathcal{L}$ is a $\lambda-$system that contains $\mathcal{P}$ then $\sigma(\mathcal{P}) \subset \mathcal{L}.$

\textbf {Theorem 2.1.7} Suppose $\mathcal{A}_1, \mathcal{A}_2,..., \mathcal{A}_n$ are independent and each $\mathcal{A}_i$ is a $\pi$-system. Then $\sigma(\mathcal{A}_1), \sigma(\mathcal{A}_2),...,\sigma(\mathcal{A}_n)$ are independent.

\textbf {Theorem 2.1.8} In order for $X_1 ,..., X_n$ to be independent, it is sufficient that for all $x_1 ,..., x_n \in (-\infty, \infty]$
\begin{center}
$P(X_1 \leq x_1 ,..., X_n \leq x_n) = \prod_{i=1}^{n} P(X_i \leq x_i)$
\end{center}

\textbf {Theorem 2.1.9} Suppose $\mathcal{F}_{i,j}, 1 \leq i \leq n, 1 \leq j \leq m(i)$ are independent and let $\mathcal{G}_i = \sigma(\cup_j \mathcal{F}_{i,j}).$ Then $\mathcal{G}_1 ,..., \mathcal{G}_n$ are independent.

\textbf {Theorem 2.1.10} If for $1 \leq i \leq n, 1 \leq j \leq m(i),$  $X_{i,j}$ are independent and $\mathcal{f}_i : R^{m(i)} \rightarrow R$ are measurable then $f_i (X_{i,1},..., X_{i,m(i)})$ are independent.

\textbf {Chapter 2.1.2 Independence, Distribution and Expectation}

\textbf {Theorem 2.1.11} Suppose $X_1 ,..., X_n$ are independent random variable and $X_i$ has distribution $\mu_i$, then ($X_1 ,..., X_n)$ has distribution $\mu_1 x \dots x \mu_n$.

\textbf {Theorem 2.1.12} Suppose X and Y are independent and have distributions $\mu$ and v. If $h: R^2 \rightarrow \mathcal{R}$ is a measurbale function with $h \geq 0$ or $E|h(X,Y)| < \infty$ then
\begin{center}
$Eh(X,Y) = \int \int h(x,y) \mu(dx) v(dy)$
\end{center}
In particular, if h(x,y) = f(x)g(y) where $f, g: R \rightarrow \mathcal{R}$ are measurable functions with f, $g \geq 0$ or $E|f(X)|$ and $E|g(Y)| < \infty$ then
\begin{center}
$Ef(X)g(Y) = Ef(X) \cdot Eg(Y)$
\end{center}

\textbf {Theorem 2.1.13} If $X_1 ... X_n$ are independent and have (a) $X_i \geq 0$ for all i, or (b) $E|X_i| < \infty$ for all i then
\begin{center}
$E(\prod_{i=1}^n X_i ) = \prod_{i=1}^n EX_i$
\end{center}
i.e. the expectation on the left exists and has the value given on the right.

\textbf {Chapter 2.1.3 Sums of Independent Random Variables} 
\textbf {Theorem 2.1.15} If X and Y are independent, $F(x) = P(X \leq x)$, and $G(y) = P(Y \leq y)$, then
\begin{center}
$P(X + Y \leq z) = \int F(z-y) dG(y)$
\end{center}
The integral on the right hand side is called the convolution of F and G and is denoted F * G(z). The meaning of dG(y) will be explained in the proof.

\textbf {Theorem 2.1.16} Suppose that X and density f and Y with distribution function G are independent. Then X + Y has density
\begin{center}
$h(x) = \int f(x-y) dG(y)$
\end{center}
When Y has density g, the last formula can be written as
\begin{center}
$h(x) = \int f(x-y) g(y) dy$
\end{center}

\textbf {Theorem 2.1.18} If X = gamma $(\alpha, \lambda)$ and Y = gamma$(\beta, \lambda)$ are independent then X + Y is gamma$(\alpha + \beta, \lambda)$. Consequently if $X_1,...,X_n$ are independent exponential$(\lambda)$ r.v.'s then $X_1 + \dots + X_n$ has a gamma$(n, \lambda)$ distribution.

\textbf {Theorem 2.1.20} If X = normal$(\mu, a)$ and Y = normal(v, b) are independent then X + Y = normal($\mu + v, a + b)$.

\textbf {Chapter 2.1.4 Constructing Independent Random Variables}

\textbf {Theorem 2.1.21 Kolmogorov's extension theorem} Suppose we are given probability measures $\mu_n$ on ($R^n, \mathcal{R}^n$) that are consistent that is,
\begin{center}
$\mu_{n+1} ((a_1 , b_1] x \dots x (a_n , b_n] x R) = \mu_n ((a_1, b_1] x \dots x (a_n, b_n])$
\end{center}
Then there is a unique probability measure P on ($R^N, \mathcal{R}^N)$ with
\begin{center}
$P(\omega : \omega_i \in (a_i, b_i], 1 \leq i \leq n) = \mu_n ((a_1, b_1] x \dots x (a_n, b_n])$
\end{center}

\textbf {Theorem 2.1.22} If S is a Borel subset of a complete seperable metric space M, and S is the collection of Borel subsets of S, then (S, $\mathcal{S})$ is nice.

\textbf {Chapter 2.2 Weak Laws of Large Numbers}

\textbf {Chapter 2.2.1 $L^2$ Weak Laws}

\textbf {Theorem 2.2.1} Let $X_1 ,..., X_n$ have $E(X_i^2) < \infty$ and be uncorrelated. Tehn
\begin{center}
$var(X_1 + \dots + X_n) = var(X_1) + \dots + var(X_n)$
\end{center}
where var(Y) = the variance of Y.

\textbf {Lemma 2.2.2} If $p > 0$ and $E|Z_n|^p \rightarrow 0$ then $Z_n \rightarrow 0$ in probability.

\textbf {Theorem 2.2.3 $L^2$ weak law} Let $X_1, X_2 ,...$ be uncorrelated random variables with $EX_i = \mu$ and $var(X_i) \leq C < \infty$. If $S_n = X_1 + \dots + X_n$ then as $n \rightarrow \infty,$ ${S_n}/n \rightarrow \mu$ in $L^2$ and in probability.

\textbf {Chapter 2.2.2 Triangular Arrays} 

\textbf {Theorem 2.2.6} Let $\mu_n = ES_n, \sigma_n^2 = var(S_n)$. If $\sigma_n^2 /b_n^2 \rightarrow 0$ then
\begin{center}
$\frac{S_n - \mu_n}{b_n} \rightarrow 0$ in probability
\end{center}

\textbf {Lemma 2.2.9} $X_{n, 1} ,..., X_{n,n}$ are independent and $P(X_{n,j} = 1) = \frac{1}{n-j+1}$

\textbf {Chapter 2.2.3 Truncation}

\textbf {Theorem 2.2.11 Weak law for triangular arrays} For each n let $X_{n,k}, 1 \leq k \leq n$ be independent. Let $b_n > 0$ with $b_n \rightarrow \infty$ and let $X_{n,k} = X_{n,k} 1(|X_{n,k}| \leq  b_n)$. Suppose that as $n \rightarrow \infty$
\begin{center}
(i) $\sum_{k=1}^n P(|X_{n,k}| > b_n) \rightarrow 0$ and \\
(ii) $b_n^{-2} \sum_{k=1}^n E X_{n,k}^2 \rightarrow 0$
\end{center}
If we let $S_n = X_{n,1} + \dots + X_{n,n}$ and put $a_n = \sum_{k=1}^n E \bar{X}_{n,k}$ then
\begin{center}
$(S_n - a_n) / b_n \rightarrow 0$ in probability
\end{center}

\textbf {Theorem 2.2.12 Weak Law of Large Numbers} Let $X_1 , X_2 \dots$ be i.i.d. with
\begin{center}
$x P(|X_i| > x) \rightarrow 0$ as $x \rightarrow \infty$
\end{center}

\textbf {Lemma 2.2.13} If $Y \geq 0$ and $p > 0$ then $E(Y^p) = \int_0^\infty py^{p-1} P(Y > y) dy$

\textbf {Theorem 2.2.14} Let $X_1 , X_2 ,...,$ be i.i.d. with $E|X_i| < \infty$. Let $S_n = X_1 + \dots + X_n$ and let $\mu = EX_1.$ Then $S_n/n \rightarrow \mu$ in probability.

\textbf {Chapter 2.3 Borel-Cantelli Lemmas}

\textbf {Theorem 2.3.1 Borel-Cantelli Lemma} If $\sum_{n=1}^\infty P(A_n) < \infty$ then 
\begin{center}
$P(A_n i.o.) = 0$
\end{center}

\textbf {Theorem 2.3.2} $X_n \rightarrow X$ in probability if and only if for every subsequence $X_{n(m)}$ there is a further subsequence $X_{n(m_k)}$ that converges almost surely to X.

\textbf {Theorem 2.3.3} Let $y_n$ be a sequence of elements of a topological space. If every subsequence $y_{n(m)}$ has a further subsequence $y_{n(m_k)}$ that converges to y then $y_n \rightarrow y$

\textbf {Theorem 2.3.4} If f is continous and $X_n \rightarrow X$ in probability then $f(X_n) \rightarrow f(X)$ in probability. If, in addition, f is bounded then $E f(X_n) \rightarrow E f(X)$.

\textbf {Theorem 2.3.5} Let $X_1, X_2 ...$ be i.i.d. with $EX_i = \mu$ and $EX_i^4 < \infty$. If $S_n = X_1 + \dots + X_n$ then $S_n / n \rightarrow \mu$ a.s.

\textbf {Theorem 2.3.7 The second Borel-Cantelli lemma} If the events $A_n$ are independent then $\sum P(A_n) = \infty$ implies $P(A_n$ i.o. ) = 1

\textbf {Theorem 2.3.8} If $X_1 , X_2 ,...,$ are i.i.d. with $E|X_i| = \infty$ then $P(|X_n| \geq n$ i.o.) = 1. So if $S_n = X_1 + \dots + X_n$ then P(lim $S_n / n$ exists $\in (-\infty, \infty)) = 0$

\textbf {Theorem 2.3.9} If $A_1 , A_2 ....$ are pairwise independent and $\sum_{n=1}^\infty P(A_n) = \infty$ then as $n \rightarrow \infty$
\begin{center}
$\sum_{m=1}^n 1 A_m / \sum_{m=1}^n P(A_m) \rightarrow 1$ a.s.
\end{center}

\textbf {Theorem 2.3.11} If $R_n = \sum_{m=1}^n 1 A_m$ is the number of records at time n then as $n \rightarrow \infty$
\begin{center}
$R_n / log n \rightarrow 1$ a.s.
\end{center}

\textbf {Chapter 2.4 Strong Law of Large Numbers}

\textbf {Theorem 2.4.1 Strong law of large numbers.} Let $X_1 , X_2 ,...,$ be pairwise independent identically distributed random variables with $E|X_i| < \infty$. Let $EX_i = \mu$ and $S_n = X_1 + \dots + X_n$. Then $S_n / n \rightarrow \mu$ a.s. as $n \rightarrow \infty$

\textbf {Lemma 2.4.2} Let $Y_k = X_k 1_{(|X_k | \leq k)}$ and $T_n = Y_1 + \dots Y + Y_n$. It is sufficient to prove that $T_n / n \rightarrow \mu$ a.s.

\textbf {Lemma 2.4.3} $\sum_{k=1}^\infty var(Y_k) / k^2 \leq 4E|X_1| < \infty.$

\textbf {Lemma 2.4.4} If $y \geq 0$ then $2y\sum_{k>y} k^{-2} \leq 4$

\textbf {Theorem 2.4.5} Let $X_1 , X_2 ,...,$ be i.i.d. with $EX_i^+ = \infty$ and $EX_i^- < \infty.$ If $S_n = X_1 + \dots + X_n$ then $S_n / n \rightarrow \infty$ a.s.

\textbf {Theorem 2.4.7} If $EX_1 = \mu \leq \infty$ then as $t \rightarrow \infty$, 
\begin{center}
$N_t / t \rightarrow 1/\mu$ \tab a.s. $(1/\infty = 0)$
\end{center}

\textbf {Theorem 2.4.9 The Glivenko-Cantelli theorem} As $n \rightarrow \infty$,
\begin{center}
$sup_{\substack{x}} |F_n(x) - F(x)| \rightarrow 0$ \tab a.s.
\end{center}

\textbf {Chapter 2.5 Convergence of Random Series*}

\textbf {Theorem 2.5.3 Komogorov's 0-1 law} If $X_1 , X_2 ,...$ are independent and $A \in \mathcal{T}$ then P(A) = 0 or 1.

\textbf {Theorem 2.5.4 Hewitt-Savage 0-1 law}. If $X_1 , X_2 ,...$ are i.i.d. and $A \in \epsilon$ then $P(A) \in \{0,1\}$

\textbf {Theorem 2.5.5 Komogorov's maximal inequality.} Suppose $X_1 ,..., X_n$ are independent with $EX_i = 0$ and var$(X_i) < \infty$. If $S_n = X_1 + \dots + X_n$ then
\begin{center}
$P(max{1 \leq k \leq n} |S_k| \geq x ) \leq x^{-2} var(S_n)$
\end{center}

\textbf {Theorem 2.5.6} Suppose $X_1 , X_2 ,...$ are independent and have $EX_n = 0$. If 
\begin{center}
$\sum_{n=1}^\infty var(X_n) < \infty$
\end{center}
then with probability one $\sum_{n=1}^\infty X_n (\omega)$ converges.

\textbf {Theorem 2.5.8 Kolmogorov's three-series theorem} Let $X_1 , X_2 ,...$ be independent. Let $A > 0$ and let $Y_i = X_i 1_{(|X_i| \leq A)}.$ \tab In order that $\sum_{n=1}^\infty X_n$ converges a.s. it is necessary and sufficient that
\begin{center}
(i) $\sum_{n=1}^\infty P(|X_n| > A) < \infty,$ (ii) $\sum_{n = 1}^\infty EY_n$ converges, and (iii) $\sum_{n=1}^\infty var(Y_n) < \infty$
\end{center} 

\textbf {Theorem 2.5.9 Kronecker's lemma} If $a_n \uparrow \infty$ and $\sum_{n=1}^\infty x_n / a_n$ converges then
\begin{center}
$a_n^{-1} \sum_{m=1}^n x_m \rightarrow 0$
\end{center}

\textbf {Theorem 2.5.10 The strong law of large numbers} Let $X_1 , X_2 ,...$ be i.i.d. random variables with $E|X_i| < \infty$. Let $EX_i = \mu$ and $S_n = X_1 + ... + X_n$. Then $S_n / n \rightarrow \mu$ a.s. as $n \rightarrow \infty$

\textbf {Chapter 2.5.1 Rates of Convergence}

\textbf {Theorem 2.5.11} Let $X_1 , X_2 ,...$ be i.i.d. random variables with $EX_i = 0$ and $EX_i^2 = \sigma^2 < \infty$. Let $S_n = X_1 + \dots + X_n$. If $\epsilon > 0$ then
\begin{center}
$S_n / n^{1/2} (log n)^{1/2+\epsilon} \rightarrow 0$ a.s.
\end{center}

\textbf {Theorem 2.5.12} Let $X_1 , X_2 ,...$ be i.i.d. with $EX_1 = 0$ and $E|X_1|^p < \infty $ where $1 < p < 2$. If $S_n = X_1 + \dots + X_n$ then $S_n /n^{1/p} \rightarrow 0$ a.s.

\textbf {Chapter 2.5.2 Infinite Mean}

\textbf {Theorem 2.5.13} Let $X_1 , X_2 ,...$ be i.i.d. with $E|X_1| = \infty$ and let $S_n = X_1 + \dots + X_n$. Let $a_n$ be a sequence of positive numbers with $a_n / n$ increasing. Then lim$sup_{n \to \infty} |S_n| / a_n = 0$ or $\infty$ according as $\sum_n P(|X_1| \geq a_n) < \infty$ or $= \infty$

\textbf {Chapter 2.6 Renewal Theorey*}

\textbf {Theorem 2.6.1} As $t \rightarrow \infty,$ $N_t / t \rightarrow 1/\mu$ a.s. where $\mu = E\xi_i \in (0,\infty]$ and $1/\infty = 0$

\textbf {Theorem 2.6.2 Wald's equation}. Let $X_1 , X_2 ,...$ be i.i.d. with $E|X_i| < \infty.$ If N is a stopping time with $EN < \infty$ then $ES_N = EX_1 EN.$

\textbf {Theorem 2.6.3} As $t \rightarrow \infty, U(t)/t \rightarrow 1/\mu$

\textbf {Theorem 2.6.4 Blackwell's renewal theorem} If F is nonarithmetic then
\begin{center}
$U([t,t + h]) \rightarrow h/\mu \tab$ as $t \rightarrow \infty$
\end{center}

\textbf {Theorem 2.6.9} If h is bounded then the function
\begin{center}
$H(t) = \int_{0}^{t} h(t-s) dU(s)$
\end{center}
is the unique solution of the renewal equation that is bounded on bounded intervals.

\textbf {Theorem 2.6.12 The renewal theorem} If F is nonarithmetic and h is directly Riemann integrable then as $t \rightarrow \infty$
\begin{center}
$H(t) \rightarrow \frac{1}{\mu} \int_{0}^{\infty} h(s) ds$
\end{center}

\textbf {Lemma 2.6.13} If $h(x) \geq 0$ is decreasing with $h(0) < \infty$ and $\int_{0}^{\infty} h(x) dx < \infty$, then h is directly Riemann integrable. 

\textbf {Chapter 2.7 Large Deviations*}

\textbf {Lemma 2.7.1} If $\gamma_{m+n} \geq \gamma_m + \gamma_n$ then as $n \rightarrow \infty, \gamma_n  / n \rightarrow sup_m \gamma_m / m$.

\textbf {Lemma 2.7.2} If $a > \mu$ and $\theta > 0$ is small then $a\theta - k(\theta) > 0$

\textbf {Theorem 2.7.7} Suppose in addition to (H1) and (H2) that there is $\theta_a \in (0, \theta_+)$ so that $a = \varphi^{'}(\theta_a) / \varphi(\theta_a).$ Then as $n \rightarrow \infty$
\begin{center}
$n^{-1} logP(S_n \geq na) \rightarrow -a\theta_a + log \varphi(\theta_a)$
\end{center}

\textbf {Lemma 2.7.8} $\frac {d F^n}{d F_\lambda^n} = e^{-\lambda x} \varphi(\lambda)^n$

\textbf {Theorem 2.7.9} Suppose $x_o = sup\{ x : F(x) < 1\} < \infty$ and F is not a point mass at $x_0. \phi(\theta) < \infty$ for all $\theta > 0$ and $\phi^{'}(\theta)/\phi(\theta) \rightarrow x_o$ as $\theta \uparrow \infty$

\textbf {Theorem 2.7.10} Suppose $x_o = \infty, \theta_+ < \infty$, and $\varphi^{'}(\theta)/\varphi(\theta)$ increases to a finite limit $a_0$ as $\theta \uparrow \theta_+.$ If $a_0 \leq a < \infty$
\begin{center}
$n^{-1} log P(S_n \geq na) \rightarrow -a\theta_+ + log \varphi(\theta_+)$
\end{center}
i.e. $\gamma(a)$ is linear for $a \geq a_0$.









\end{document}